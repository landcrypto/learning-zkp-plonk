\hypertarget{ux7406ux89e3-plonkux4e94ux591aux9879ux5f0fux627fux8bfa}{%
\chapter{多项式承诺}\label{ux7406ux89e3-plonkux4e94ux591aux9879ux5f0fux627fux8bfa}}

\hypertarget{ux4ec0ux4e48ux662fux591aux9879ux5f0fux627fux8bfa}{%
\section{什么是多项式承诺}\label{ux4ec0ux4e48ux662fux591aux9879ux5f0fux627fux8bfa}}

所谓承诺,是对消息「锁定」,得到一个锁定值。这个值被称为对象的「承诺」。

\[
c = commit(x)
\]

这个值和原对象存在两个关系,即 Hiding 与 Binding。

Hiding: \(c\) 不暴露任何关于 \(x\) 的信息;

Binding:难以找到一个 \(x', x'\neq x\),使得 \(c=commit(x')\)。

最简单的承诺操作就是 Hash 运算。请注意这里的 Hash
运算需要具备密码学安全强度,比如 SHA256, Keccak 等。除了 Hash
算法之外,还有 Pedersen 承诺等。

顾名思义,多项式承诺可以理解为「多项式」的「承诺」。如果我们把一个多项式表达成如下的公式,

\[
f(X) = a_0 + a_1X + a_2X^2 + \cdots + a_nX^n
\]

那么我们可以用所有系数构成的向量来唯一标识多项式 \(f(X)\)。

\[
(a_0, a_1, a_2,\ldots, a_n)
\]

如何对一个多项式进行承诺?很容易能想到,我们可以把「系数向量」进行 Hash
运算,得到一个数值,就能建立与这个多项式之间唯一的绑定关系。

\[
C_1 = \textrm{SHA256}(a_0\parallel a_1 \parallel a_2 \parallel \cdots \parallel a_n)
\]

或者,我们也可以使用 Petersen 承诺,通过一组随机选择的基,来计算一个 ECC
点:

\[
C_2 = a_0 G_0 + a_1  G_1 + \cdots + a_n G_n
\]

如果在 Prover 承诺多项式之后,Verifier
可以根据这个承诺,对被锁定的多项式进行求值,并希望 Prover
可以证明求值的正确性。假设 \(C=Commit(f(X))\),Verifier 可以向提供承诺的
Prover 询问多项式在 \(X=\zeta\) 处的取值。Prover
除了回复一个计算结果之外(如 \(f(\zeta) = y\)) ,还能提供一个证明
\(\pi\),证明 \(C\) 所对应的多项式 \(f(X)\) 在 \(X=\zeta\) 处的取值
\(y\) 的正确性。

多项式承诺的这个「携带证明的求值」特性非常有用,它可以被看成是一种轻量级的「可验证计算」。即
Verifier 需要把多项式 \(f(X)\)
的运算代理给一个远程的机器(Prover),然后验证计算(计算量要小于直接计算\(f(X)\))结果
\(y\)
的正确性;多项式承诺还能用来证明秘密数据(来自Prover)的性质,比如满足某个多项式,Prover
可以在不泄漏隐私的情况下向 Verifier 证明这个性质。

虽然这种可验证计算只是局限在多项式运算上,而非通用计算。但通用计算可以通过各种方式转换成多项式计算,从而依托多项式承诺来最终实现通用的可验证计算。

按上面 \(C_2\) 的方式对多项式的系数进行 Pedersen 承诺,我们仍然可以利用
Bulletproof-IPA
协议来实现求值证明,进而实现另一种多项式承诺方案。此外,还有 KZG10
方案,FRI,Dark,Dory 等等其它方案。

\hypertarget{kzg10-ux6784ux9020}{%
\section{KZG10 构造}\label{kzg10-ux6784ux9020}}

与 Pedersen 承诺中用的随机基向量相比,KZG10
多项式承诺需要用一组具有内部代数结构的基向量来代替。

\[
(G_0, G_1, G_2, \ldots, G_{d-1}, H_0, H_1) = (G, \chi G, \chi^2G, \ldots, \chi^{d-1}G, H, \chi H)
\]

请注意,这里的 \(\chi\) 是一个可信第三方提供的随机数,也被称为
Trapdoor,需要在第三方完成 Setup 后被彻底删除。它既不能让 Verifier
知道,也不能让 Prover 知道。当 \(\vec{G}\) 设置好之后, \(\chi\)
被埋入了基向量中。这样一来,从外部看,这组基向量与随机基向量难以被区分。其中
\(G\in\mathbb{G}_1\),而 \(H\in\mathbb{G}_2\),并且存在双线性映射
\(e\in \mathbb{G}_1\times\mathbb{G}_2\to \mathbb{G}_T\)。

对于一个多项式 \(f(X)\) 进行 KZG10 承诺,也是对其系数向量进行承诺:

\[
\begin{split}
C_{f(X)} &= a_0 G_0 + a_1  G_1 + \cdots + a_{n-1} G_{n-1} \\
 & = a_0  G + a_1 \chi G + \cdots + a_{n-1}\chi^{n-1} G\\
 & = f(\chi) G
\end{split}
\]

这样承诺 \(C_{f(X)}\) 巧好等于 \(f(\chi) G\)。

对于双线性群,我们下面使用 Groth 发明的符号 \([1]_1\triangleq G\),
\([1]_2\triangleq H\) 表示两个群上的生成元,这样 KZG10
的系统参数(也被称为 SRS, Structured Reference String)可以表示如下:

\[
\mathsf{srs}=([1]_1,[\chi]_1,[\chi^2]_1,[\chi^3]_1,\ldots,[\chi^{n-1}]_1,[1]_2,[\chi]_2)
\]

而 \(C_{f(X)}=[f(\chi)]_1\)。

下面构造一个 \(f(\zeta) = y\) 的 Open
证明。根据多项式余数定理,我们可以得到下面的等式:

\[
f(X) = q(X)\cdot (X-\zeta) + y
\]

这个等式可以解释为,任何一个多项式都可以除以另一个多项式,得到一个商多项式加上一个余数多项式。由于多项式在
\(X=\zeta\) 处的取值为 \(y\),那么我们可以确定:余数多项式一定为 \(y\)
,因为等式右边的第一项在 \(X=\zeta\) 处取值为零。所以,如果
\(f(\zeta)=y\),我们可以断定: \(g(X) = f(X)-y\) 在 \(X=\zeta\)
处等零,所以 \(\zeta\) 为 \(g(X)\) 的根,于是 \(g(X)\) 一定可以被
\((X-\zeta)\) 这个不可约多项式整除,即一定\textbf{存在}一个商多项式
\(q(X)\),满足上述等式。

而 Prover 则可以提供 \(q(X)\) 多项式的承诺,记为 \(C_q\),作为
\(f(\zeta)=y\) 的证明,Verifier 可以检查 \([q(\chi)]\)
是否满足整除性来验证证明。因为如果 \(f(\zeta)\neq y\),那么 \(g(X)\)
则无法被 \((X-\zeta)\) 整除,即使 Prover
提供的承诺将无法通过整除性检查:

\[
(f(X)-y)\cdot 1 \overset{?}{=} q(X) \cdot (X-x)
\]

承诺 \(C_{f(X)}\) 是群 \(\mathbb{G}_1\)
上的一个元素,通过承诺的加法同态映射关系,以及双线性映射关系
\(e\in \mathbb{G}_1\times\mathbb{G}_2\to \mathbb{G}_T\),Verifier 可以在
\(\mathbb{G}_T\) 上验证整除性关系:

\[
e(C\_{f(X)} - y[1]_1, [1]_2) \overset{?}{=} e(C\_{q(X)}, [\chi]_2 - \zeta [1]_2)
\]

有时为了减少 Verifier 在 \(\mathbb{G}_2\)
上的昂贵操作,上面的验证等式可以变形为:

\[
f(X) + \zeta\cdot q(X) - y =  q(X)\cdot X
\]

\[
e(C\_{f(X)} + \zeta\cdot C\_{q(X)} -y, [1]_2)\overset{?}{=} e(C\_{q(X)}, [\chi]_2)
\]

\hypertarget{ux540cux70b9-open-ux7684ux8bc1ux660eux805aux5408}{%
\section{同点 Open
的证明聚合}\label{ux540cux70b9-open-ux7684ux8bc1ux660eux805aux5408}}

在一个更大的安全协议中,假如同时使用多个多项式承诺,那么他们的 Open
操作可以合并在一起完成。即把多个多项式先合并成一个更大的多项式,然后仅通过
Open 一点,来完成对原始多项式的批量验证。

假设我们有多个多项式, \(f_1(X)\), \(f_2(X)\),Prover 要同时向 Verifier
证明 \(f_1(\zeta)=y_1\) 和 \(f_2(\zeta)=y_2\),那么有

\[
\begin{array}{l}
f_1(X) = q_1(X)\cdot (X-\zeta) + y_1\\ 
f_2(X) = q_2(X) \cdot (X-\zeta) + y_2 \\
\end{array}
\]

通过一个随机数 \(\nu\),Prover 可以把两个多项式 \(f_1(X)\) 与 \(f_2(X)\)
折叠在一起,得到一个临时的多项式 \(g(X)\) :

\[
g(X) = f_1(X) + \nu\cdot f_2(X)
\]

进而我们可以根据多项式余数定理,推导验证下面的等式:

\[
g(X) - (y_1 + \nu\cdot y_2) = (X-\zeta)\cdot (q_1(X) + \nu\cdot q_2(X))
\]

我们把等号右边的第二项看作为「商多项式」,记为 \(q(X)\):

\[
q(X) = q_1(X) + \nu\cdot q_2(X)
\]

假如 \(f_1(X)\) 在 \(X=\zeta\) 处的求值证明为 \(\pi_1\),而 \(f_2(X)\)
在 \(X=\zeta\) 处的求值证明为 \(\pi_2\),那么根据群加法的同态性,Prover
可以得到商多项式 \(q(X)\) 的承诺:

\[
[q(\chi)]_1 = \pi = \pi_1 + \nu\cdot\pi_2
\]

因此,只要 Verifier 发给 Prover 一个额外的随机数
\(\nu\),双方就可以把两个(甚至多个)多项式承诺折叠成一个多项式承诺
\(C_g\):

\[
C_g = C_1 + \nu\ast C_2
\]

并用这个折叠后的 \(C_g\) 来验证多个多项式在一个点处的运算取值:

\[
y_g = y_1 + \nu\cdot y_2
\]

从而把多个求值证明相应地折叠成一个,Verifier 可以一次验证完毕:

\[
e(C-y\ast G_0, H_0) \overset{?}{=}e(\pi, H_1 - x\ast H_0)
\]

由于引入了随机数
\(\nu\),因此多项式的合并不会影响承诺的绑定关系(Schwartz-Zippel
定理)。

\hypertarget{ux534fux8bae}{%
\subsection{协议:}\label{ux534fux8bae}}

公共输入: \(C\_{f_1}=[f_1(\chi)]_1\), \(C\_{f_2}=[f_2(\chi)]_1\),
\(\zeta\), \(y_1\), \(y_2\)

私有输入: \(f_1(X)\), \(f_2(X)\)

证明目标: \(f_1(\zeta)=y_1\), \(f_2(\zeta)=y_2\)

第一轮:Verifier 提出挑战数 \(\nu\)

第二轮:Prover 计算 \(q(X)=f_1(X)+\nu\cdot f_2(X)\),并发送
\(\pi=[q(\chi)]_1\)

第三轮:Verifier 计算 \(C_g=C_{f_1} + \nu\cdot C_{f_2}\),
\(y_g = y_1 + \nu\cdot y_2\)

\[
e(C_g - [y_g]_1, [1]_2)\overset{?}{=}e(\pi, [\chi-\zeta]_2)
\]

\hypertarget{ux591aux9879ux5f0fux7ea6ux675fux4e0eux7ebfux6027ux5316}{%
\section{多项式约束与线性化}\label{ux591aux9879ux5f0fux7ea6ux675fux4e0eux7ebfux6027ux5316}}

假设 \([f(\chi)]_1, [g(\chi)]_1, [h(\chi)]_1\) 分别是 \(f(X),g(X),h(X)\)
的 KZG10 承诺,如果 Verifier 要验证下面的多项式约束:

\[
f(X) + g(X) \overset{?}{=} h(X)
\]

那么 Verifier 只需要把前两者的承诺相加,然后判断是否等于 \([h(\chi)]_1\)
即可

\[
[f(\chi)]_1 + [g(\chi)]_1 \overset{?}{=} [h(\chi)]_1
\]

如果 Verifier 需要验证的多项式关系涉及到乘法,比如:

\[
f(X) \cdot g(X) \overset{?}{=} h(X)
\]

最直接的方法是利用双线性群的特性,在 \(\mathbb{G}_T\)
上检查乘法关系,即验证下面的等式:

\[
e([f(\chi)]_1, [g(\chi)]_2) \overset{?}{=} e([h(\chi)]_1, [1]_2)
\]

但是如果 Verifier 只有 \(g(X)\) 在 \(\mathbb{G}_1\) 上的承诺
\([g(\chi)]_1\),而非是在 \(\mathbb{G}_2\) 上的承诺
\([g(\chi)]_2\),那么Verifer 就无法利用双线性配对操作来完成乘法检验。

另一个直接的方案是把三个多项式在同一个挑战点 \(X=\zeta\)
上打开,然后验证打开值之间的关系是否满足乘法约束:

\[
f(\zeta)\cdot g(\zeta)\overset{?}{=} h(\zeta)
\]

同时 Prover 还要提供三个多项式求值的证明
\((\pi_{f(\zeta)},\pi_{g(\zeta)},\pi_{h(\zeta)})\) 供 Verifier 验证。

这个方案的优势在于多项式的约束关系可以更加复杂和灵活,比如验证下面的稍微复杂些的多项式约束:

\[
f_1(X)f_2(X) + h_1(X)h_2(X)h_3(X) + g(X) = 0
\]

假设 Verifier 已拥有这些多项式的 KZG10 承诺, \([f_1(\chi)]_1\),
\([f_2(\chi)]_1\), \([h_1(\chi)]_1\), \([h_2(\chi)]_1\),
\([h_3(\chi)]_1\), \([g(\chi)]_1\)。最直接粗暴的方案是让 Prover
在挑战点 \(X=\zeta\) 处打开这 6 个承诺,发送 6 个 Open
值和对应的求值证明:

\[
(f_1(\zeta),\pi_{f_1}),(f_2(\zeta),\pi_{f_2}),(h_1(\zeta),\pi_{h_1}),(h_2(\zeta),\pi_{h_2}),(h_3(\zeta),\pi_{h_3}),(g(\zeta),\pi_{g})
\]

Verifier 验证 \(6\) 个求值证明,并且验证多项式约束:

\[
f_1(\zeta)f_2(\zeta) + h_1(\zeta)h_2(\zeta)h_3(\zeta) + g(\zeta) \overset{?}{=} 0
\]

我们可以进一步优化,比如考虑对于 \(f(X) \cdot g(X) = h(X)\)
这样一个简单的多项式约束,Prover 可以减少 Open 的数量。比如 Prover 先
Open \(\bar{f} = f(\zeta)\),发送求值证明 \(\pi\_{f(\zeta)}\)
然后引入一个辅助多项式 \(L(X)= \bar{f}\cdot g(X)-h(X)\),再 Open
\(L(X)\) 在 \(X=\zeta\) 处的取值。

显然对于一个诚实的 Prover, \(L(\zeta)\) 求值应该等于零。对于
Verifier,它在收到 \(\bar{f}\)
之后,就可以利用承诺的加法同态性,直接构造 \(L(X)\) 的承诺:

\[
[L(\chi)]_1 = \bar{f}\cdot [g(\chi)]_1 - [h(\chi)]_1
\]

这样一来,Verifier 就不需要单独让 Prover 发送 \(L(X)\) 的
Opening,也不需要发送新多项式 \(L(X)\) 的承诺。Verifier 然后就可以验证
\(f(X) \cdot g(X) = h(X)\) 这个多项式约束关系:

\[
e([L(\chi)]_1, [1]_2)\overset{?}{=} e(\pi\_{L(\zeta)}, [\chi-\zeta]_2)
\]

这个优化过后的方案,Prover 只需要 Open 两次。第一个 Opening 为
\(\bar{f}\),第二个 Opening 为 \(0\)。而后者是个常数,不需要发送给
Verifier。Prover
只需要发送两个求值证明,不过我们仍然可以用上一节提供的聚合证明的方法,通过一个挑战数
\(\nu\),Prover 可以聚合两个多项式承诺,然后仅需要发送一个求值证明。

我们下面尝试优化下 \(6\) 个多项式的约束关系的协议:
\(f_1(X)f_2(X) + h_1(X)h_2(X)h_3(X) + g(X) = 0\)。

\hypertarget{ux534fux8bae-1}{%
\subsection{协议:}\label{ux534fux8bae-1}}

公共输入: \(C\_{f_1}=[f_1(\chi)]_1\), \(C\_{f_2}=[f_2(\chi)]_1\),
\(C\_{h_1}=[h_1(\chi)]_1\), \(C\_{h_2}=[h_2(\chi)]_1\),
\(C\_{h_3}=[h_3(\chi)]_1\), \(C\_{g}=[g(\chi)]_1\),

私有输入: \(f_1(X)\), \(f_2(X)\), \(h_1(X)\), \(h_2(X)\),
\(h_3(X)\), \(g(X)\)

证明目标: \(f_1(X)f_2(X) + h_1(X)h_2(X)h_3(X) + g(X) = 0\)

第一轮:Verifier 发送 \(X=\zeta\)

第二轮:Prover 计算并发送三个Opening, \(\bar{f_1}=f_1(\zeta)\),
\(\bar{h}_1=h_1(\zeta)\), \(\bar{h}_2=h_2(\zeta)\),

第三轮:Verifier 发送 \(\nu\) 随机数

第四轮:Prover 计算 \(L(X)\) ,利用 \(\nu\) 折叠
\((L(X), f_1(X),h_1(X),h_2(X))\) 这四个承诺,并计算商多项式
\(q(X)\),发送其承诺 \([q(\chi)]_1\) 作为折叠后的多项式在 \(X=\zeta\)
处的求值证明

\[
L(X)=\bar{f}_1\cdot f_2(X) + \bar{h}_1\bar{h}_2\cdot h_3(X) + g(X)
\]

\[
q(X)=\frac{1}{X-\zeta}\Big(L(X) + \nu\cdot (f_1(X)-\bar{f}_1)+\nu^2\cdot (h_1(X)-\bar{h}_1)+\nu^3\cdot (h_2(X)-\bar{h}_2)\Big)
\]

第五轮:Verifier 计算辅助多项式 \(L(X)\) 的承诺 \([L]_1\):

\[
[L]_1 = \bar{f}_1\cdot[f_2(\chi)]_1 + \bar{h}_1\bar{h}_2\cdot[h_3(\chi)]_1 + [g(\chi)]_1
\]

计算折叠后的多项式的承诺:

\[
[F]_1=[L]_1 + \nu \cdot  [f_1(\chi)]_1+\nu^2[h_1(\chi)]_1+\nu^3[h_2(\chi)]_1
\]

计算折叠后的多项式在 \(X=\zeta\) 处的求值:

\[
E=\nu\cdot \bar{f}_1 + \nu^2\cdot\bar{h}_1+ \nu^3\cdot\bar{h}_2
\]

检查下面的验证等式:

\[
e([F]_1-[E]_1 + \zeta[q(\chi)]_1, [1]_2)\overset{?}{=}e([q(\chi)]_1, [\chi]_2)
\]

这个优化后的协议,Prover 仅需要发送三个
Opening,一个求值证明;相比原始方案的 6 个 Opening和 6
个求值证明,大大减小了通信量(即证明大小)。




