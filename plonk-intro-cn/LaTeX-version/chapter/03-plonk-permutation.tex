\hypertarget{ux7406ux89e3-plonkux4e09ux7f6eux6362ux8bc1ux660e}{%
\chapter{置换证明}\label{ux7406ux89e3-plonkux4e09ux7f6eux6362ux8bc1ux660e}} 

Plonkish 电路编码用两个矩阵 \((Q,\sigma)\) 描述电路的空白结构,其中
\(Q\) 为运算开关, \(\sigma\) 为置换关系,用来约束 \(W\)
矩阵中的某些位置必须被填入相等的值。本文重点讲解置换证明(Permutation
Argument)的原理。 \footnote{[WIP] Copy constraint for arbitrary number of wires. \url{https://hackmd.io/CfFCbA0TTJ6X08vHg0-9_g}} 
\footnote{Alin Tomescu. Feist-Khovratovich technique for computing KZG proofs fast. \url{https://alinush.github.io/2021/06/17/Feist-Khovratovich-technique-for-computing-KZG-proofs-fast.html}}
\footnote{Ariel Gabizon. Multiset checks in PLONK and Plookup. \url{https://hackmd.io/@arielg/ByFgSDA7D}}

\hypertarget{ux56deux987eux62f7ux8d1dux5173ux7cfb}{%
\section{回顾拷贝关系}\label{ux56deux987eux62f7ux8d1dux5173ux7cfb}}

回顾一下 Plonkish 的 \(W\) 表格,总共有三列,行数按照 \(2^2\) 对齐。

\[
\begin{array}{c|c|c|c|}
i & w_{a,i} & w_{b,i} & w_{c,i}  \\
\hline
1 & {\color{red}x_6} & {\color{blue}x_5} & {\color{green}out} \\
2 & x_1 & x_2 & {\color{red}x_6} \\
3 & x_3 & x_4 & {\color{blue}x_5} \\
4 & 0 & 0 & {\color{green}out} \\
\end{array}
\]

我们想约束 Prover 在填写 \(W\) 表时,满足下面的拷贝关系:
\(w_{a,1}=w_{c,2}\) \(w_{b,1}=w_{c,3}\) 与
\(w_{c,1}=w_{c,4}\),换句话说, \(w_{a,1}\) 位置上的值需要被拷贝到
\(w_{c,2}\) 处,而 \(w_{b,1}\) 位置上的值需要被拷贝到 \(w_{c,3}\) 处,
\(w_{c,1}\) 位置上的值被拷贝到 \(w_{c,4}\) 处。

问题的挑战性在于,Verifier 要仅通过一次随机挑战就能完成 \(W\)
表格中多个拷贝关系的证明,并且在看不到 \(W\) 表格的情况下。

Plonk 的「拷贝约束」是通过「置换证明」(Permutation
Argument)来实现,即把表格中需要约束相等的那些值进行循环换位,然后证明换位后的表格和原来的表格完全相等。

简化一下问题:如何证明两个等长向量 \(\vec{a}\) 和 \(\vec{a}'\)
满足一个已知的置换 \(\sigma\),并且 \(\vec{a}=\vec{a}'\)

\[
a_i=a'_{\sigma(i)}
\]

举一个例子,假设 \(\vec{a}=(a_0,a_1,a_2,a_3)\),
\(\vec{a}'=(a_1,a_2,a_3,a_0)\),即他们满足一个「左移循环换位」的置换关系,那么
\(\sigma=\\{0\to 1; 1\to 2; 2\to 3; 3\to0\\}\)。如何能证明
\(\vec{a}=\vec{a}'\) ,那么两个向量对应位置的值都应该相等,

\[
\begin{array}{c{|}c|c|c|c|c}
\vec{a} & a_0 & a_1 & a_2 & a_3 \\
\hline
\vec{a}' & a_1 & a_2 & a_3 & a_0 \\
\end{array}
\]

那么 \(a_0=a_1\), \(a_1=a_2\), \(a_2=a_3\),
\(a_3=a_0\),于是可以得出结论: \(a_0=a_1=a_2=a_3\),即 \(\vec{a}\)
中的全部元素都相等。

对于 \(W\) ,我们只需要针对那些需要相等的位置进行循环换位,然后让 Prover
证明 \(W\) 和经过循环换位后的 \(W'\)
表格相等,那么可实现拷贝约束。证明两个表格相等,这个可以通过多项式编码,然后进行概率检验的方式完成。剩下的工作就是如何让
Prover 证明 \(W'\) 确实是(诚实地)按照事先约定的方式进行循环移位。

那么接下来就是理解如何让 Prover 证明两个向量之间满足某一个「置换关系」。
置换证明(Permutation Argument)是 Plonk
协议中的核心部分,为了解释它的工作原理,我们先从一个基础协议开始------连乘证明(Grand
Product Argument)。

\hypertarget{ux51b7ux542fux52a8grand-product}{%
\section{冷启动:Grand
Product}\label{ux51b7ux542fux52a8grand-product}}

假设我们要证明下面的「连乘关系」 :

\[
p = q_0\cdot q_1 \cdot q_2 \cdot \cdots \cdot q_{n-2}
\]

我们在上一篇文章介绍了如何证明一组「单乘法」,通过多项式编码,把多个单乘法压缩成单次乘法的验证。

这里对付连乘的基本思路是:让 Prover
利用一组单乘的证明来实现多个数的连乘证明,然后再通过多项式的编码,交给
Verifier 进行概率检查。

强调下:思路中的关键点是如何把一个连乘计算转换成多次的单乘计算。

我们需要通过引入一个「辅助向量」,把「连乘」的计算看成是一步步的单乘计算,然后辅助向量表示每次单乘之后的「中间值」:

\[
\begin{array}{c|c|l}
q_i & r_i & \ \ q_i\cdot r_i \\
\hline
q_0 & r_0=1  & r_1=q_0\\
q_1 & r_1 & r_2=q_0\cdot q_1\\
q_2 & r_2 & r_3=q_0\cdot q_1\cdot q_2\\
\vdots & \vdots & \vdots\\
q_{n-2} & r_{n-2} & r_{n-1} = p\\
\end{array}
\]

上面表格表述了连乘过程的计算轨迹(Trace),每一行代表一次单乘,顺序从上往下计算,最后一行计算出最终的结果。

表格的最左列为要进行连乘的向量 \({q_i}\),中间列 \({r_i}\)
为引入的辅助变量,记录每次「单乘之前」的中间值,最右列表示每次「单乘之后」的中间值。

不难发现,「中间列」向量 \(\vec{r}\)
向上挪一行与「最右列」几乎一致,除了最后一个元素。该向量的第一个元素用了常数
\(1\) 作为计算初始值,「最右列」最后一个向量元素为计算结果。

向量 \(\vec{r}\) 是一个
Accumulator,即记录连乘计算过程中的每一个中间结果:

\[
r_k = \prod_{i=0}^{k-1}q_i
\]

那么显然我们可以得到下面的递归式:

\[
r_0 = 1, \qquad r_{k+1}=q_{k}\cdot r_{k}
\]

于是,表格的三列编码后的多项式也将满足下面三个约束。第一个是初始值为
\(1\):

\[
L_0(X)\cdot(r(X)-1)=0, \qquad \forall X\in H 
\]

第二个约束为递归的乘法关系:

\[
q(X)\cdot r(X) = r(\omega\cdot X), \qquad \forall X\in H\backslash{\omega^{-1}}
\]

第三个约束最后结果 \(r_{n-1}=p\):

\[
L_{n-1}(X)\cdot(r(X)-p)=0, \qquad \forall X\in H
\]

我们可以用一个小技巧来简化上面的三个约束。我们把计算连乘的表格添加一行,令
\(q_{n-1}=1/p\)(注意: \(p\) 为 \(\vec{q}\) 向量的连乘积)

\[
\begin{array}{c|c|c}
q_i & r_i & q_i\cdot r_i \\
\hline
q_0 & 1  & r_0\\
q_1 & r_0 & r_1\\
q_2 & r_1 & r_2\\
\vdots & \vdots & \vdots\\
q_{n-2} & r_{n-2} & r_{n-1}\\
q_{n-1}=\frac{1}{p} & r_{n-1} & 1 \\
\end{array}
\]

这样一来, \(r_n=r_0=1\) 。最右列恰好是 \(\vec{r}\)
的循环移位。并且上面表格的每一行都满足「乘法关系」!于是,我们可以用下面的多项式约束来表示递归的连乘:

\[
q(X)\cdot r(X)=r(\omega\cdot X), \qquad \forall X\in H
\]

接下来,Verifier 可以挑战下面的多项式等式:

\[
L_0(X)\cdot(r(X)-1)+\alpha\cdot(q(X)\cdot r(X)-r(\omega\cdot X))=h(X)\cdot z_H(X)
\]

其中 \(\alpha\) 是用来聚合多个多项式约束的随机挑战数。其中 \(h(X)\)
为商多项式, \(z_H(X)=(X-1)(X-\omega)\cdots(X-\omega^{n-1})\)。

接下来,通过 Schwartz-Zippel 定理,Verifier 可以给出挑战数 \(\zeta\)
来验证上述多项式等式是否成立。

到此为止,如果我们已经理解了如何证明一个向量元素的连乘,那么接下来的问题是如何利用「连乘证明」来实现「Multiset
等价证明」(Multiset Equality Argument)。

\hypertarget{ux4ece-grand-product-ux5230-multiset-ux7b49ux4ef7}{%
\section{从 Grand Product 到 Multiset
等价}\label{ux4ece-grand-product-ux5230-multiset-ux7b49ux4ef7}}

假设有两个向量,其中一个向量是另一个向量的乱序重排,那么如何证明它们在集合意义(注意:集合无序)上的等价呢?最直接的做法是依次枚举其中一个向量中的每个元素,并证明该元素属于另一个向量。但这个方法有个限制,就是无法处理向量会中出现两个相同元素的情况,也即不支持「多重集合」(Multiset)的判等。例如
\({1,1,2}\) 就属于一个多重集合(Multiset),那么它显然不等于
\({1, 2, 2}\),也不等于 \({2,1}\)。

另一个直接的想法是将两个向量中的所有元素都连乘起来,然后判断两个向量的连乘值是否相等。但这个方法同样有一个严重的限制,就是向量元素必须都为素数,比如
\(3\cdot6=9\cdot2\) ,但 \({3,6}\neq{9,2}\)。

修改下这个方法,我们假设向量 \({q_i}\) 为一个多项式 \(q(X)\)
的根集合,即对向量中的任何一个元素 \(q_i\),都满足
\(q(r_i)=0\)。这个多项式可以定义为:

\[
q(X) = (X-q_0)(X-q_1)(X-q_2)\cdots (X-q_{n-1})
\]

如果存在另一个多项式 \(p(X)\) 等于
\(q(X)\),那么它们一定具有相同的根集合 \({q_i}\)。比如

\[
\prod_{i}(X - q_i) = q(X) = p(X) = \prod_{i}(X - p_i)
\]

那么

\[
\\{q_i\\}=_{multiset}\\{p_i\\}
\]

我们可以利用 Schwartz-Zippel 定理来进一步地检验:向 Verifier
索要一个随机数 \(\gamma\),那么 Prover 就可以通过下面的等式证明两个向量
\({p_i}\) 与 \({q_i}\) 在多重集合意义上等价:

\[
\prod_{{i\in[n]}}(\gamma-p_i)=\prod_{i\in[n]}(\gamma-q_i)
\]

还没结束,我们需要用上一节的连乘证明方案来继续完成验证,即通过构造辅助向量(作为一个累积器),把连乘转换成多个单乘来完成证明。需要注意的是,这里的两个连乘可以合并为一个连乘,即上面的连乘相等可以转换为

\[
\prod_{{i\in[n]}}\frac{(\gamma-p_i)}{(\gamma-q_i)}=1
\]

到这里,我们已经明白如何证明「Multiset
等价」,下一步我们将完成构造「置换证明」(Permutation
Argument),用来实现协议所需的「Copy Constraints」。

\hypertarget{ux4ece-multiset-ux7b49ux4ef7ux5230ux7f6eux6362ux8bc1ux660e}{%
\section{从 Multiset
等价到置换证明}\label{ux4ece-multiset-ux7b49ux4ef7ux5230ux7f6eux6362ux8bc1ux660e}}

Multiset 等价可以被看作是一类特殊的置换证明。即两个向量 \({p_i}\) 和
\({q_i}\)存在一个「未知」的置换关系。

而我们需要的是一个支持「已知」的特定置换关系的证明和验证。也就是对一个有序的向量进行一个「公开特定的重新排列」。

先简化下问题,假如我们想让 Prover 证明两个向量满足一个奇偶位互换的置换:

\[
\begin{array}{rcl}
\vec{a} &=& (a_0, a_1, a_2, a_3,\ldots, a_{n-1}, a_n) \\
\vec{b} &=& (a_1, a_0, a_3, a_2, \ldots, a_n, a_{n-1})\\
\end{array}
\]

我们仍然采用「多项式编码」的方式把上面两个向量编码为两个多项式,
\(a(X)\) 与
\(b(X)\)。思考一下,我们可以用下面的「位置向量」来表示「奇偶互换」:

\[
\vec{i}=(1,2,3,4,\ldots, n-1, n),\quad \sigma = (2, 1, 4, 3,\ldots, n, n-1)
\]

我们进一步把这个位置向量和 \(\vec{a}\) 与 \(\vec{b}\) 并排放在一起:

\[
\begin{array}{|c|c | c|c|}
a_i & {i} & b_i & \sigma({i}) \\
\hline
a_0 & 0 & b_0=a_1 & 1 \\
a_1 & 1 & b_1=a_0 & 0 \\
a_2 & 2 & b_2=a_3 & 3 \\
a_3 & 3 & b_3=a_2 & 2 \\
\vdots & \vdots & \vdots & \vdots \\
a_n & n & b_n=a_{n-1} & n-1 \\
a_{n-1} & n-1 & b_{n-1}=a_{n} & n \\
\end{array}
\]

接下来,我们要把上表的左边两列,还有右边两列分别「折叠」在一起。换句话说,我们把
\((a_i, i)\) 视为一列元素,把 \((b_i, \sigma(i))\)
视为一个元素,这样上面表格就变成了:

\[
\begin{array}{|c|c|}
a'_i=(a_i, i) & b'_i=({b}_i, \sigma(i)) \\
\hline
(a_0, 0) & (b_0=a_1, 1) \\
(a_1, 1) & (b_1=a_0, 0) \\
\vdots & \vdots \\
(a\_{n-1}, n-1) & (b\_{n-1}=a\_{n}, n) \\
(a\_n, n) & (b\_n=a\_{n-1}, n-1) \\
\end{array}
\]

容易看出,如果两个向量 \(\vec{a}\) 与 \(\vec{b}\) 满足 \(\sigma\)
置换,那么,合并后的两个向量 \(\vec{a}'\) 和 \(\vec{b}'\) 将满足
Multiset 等价关系。

也就是说,通过把向量和位置值合并,就能够把一个「置换证明」转换成一个「多重集合等价证明」,即不用再针对某个特定的「置换关系」进行证明。

这里又出现一个问题,表格的左右两列中的元素为二元组(Pair),二元组无法作为一个「一元多项式」的根集合。

我们再使用一个技巧:再向 Verifier 索取一个随机数
\(\beta\),把一个元组「折叠」成一个值:

\[
\begin{array}{|c|c|}
a'_i=(a_i+\beta\cdot i) & b_i'=(b + \beta\cdot \sigma(i)) \\
\hline
(a_0 + \beta\cdot 0) & (b_0 + \beta\cdot 1) \\
(a_1 + \beta\cdot 1) & (b_1 + \beta\cdot 0) \\
\vdots & \vdots \\
(a\_{n-1} + \beta\cdot n-1) & (b\_{n-1} + \beta\cdot n) \\
(a\_n + \beta\cdot n) & (b\_n + \beta\cdot (n-1))\\
\end{array}
\]

接下来,Prover 可以对 \(\vec{a}'\) 与 \(\vec{b}'\) 两个向量进行 Multiset
等价证明,从而可以证明它们的置换关系。

\hypertarget{ux5b8cux6574ux7684ux7f6eux6362ux534fux8bae}{%
\section{完整的置换协议}\label{ux5b8cux6574ux7684ux7f6eux6362ux534fux8bae}}

公共输入:置换关系 \(\sigma\);

秘密输入:两个向量 \(\vec{a}\) 与 \(\vec{b}\) ;

预处理:Prover 和 Verifier 构造 \(id(X)\) 与 \(\sigma(X)\),
第一步:Prover 构造并发送 \([a(X)]\) 与 \([b(X)]\),

第二步:Verifier 发送挑战数 \(\beta\) 与 \(\gamma\),

第三步:Prover 构造辅助向量 \(\vec{z}\),

\[
\begin{split}
z_0 &= 1 \\
z_{i+1} &= z_i\cdot \frac{a_i+\beta\cdot i + \gamma}{b_i+\beta\cdot \sigma(i) + \gamma}
\end{split}
\]

构造多项式 \(z(X)\) 并发送 \([z(X)]\);

第四步:Verifier 发送挑战数 \(\alpha\);

第五步:Prover 构造 \(f(X)\) 与 \(q(X)\),并发送 \([q(X)]\)

\[
f(X)= L_0(X)(z(X)-1) + \alpha\cdot (z(\omega\cdot X)(b(X)+\beta\cdot\sigma(X)+\gamma)-z(X)(a(X)+\beta\cdot id(X)+\gamma)) 
\]

\[
q(X) = \frac{f(X)}{z_H(X)}
\]

第四步:Verifier 向 \([a(X)],[b(X)],[z(X)]\) 查询 发送 \(\zeta\),得到
\(a(\zeta)\), \(b(\zeta)\), \(z(\zeta)\), \(id(\zeta)\) 与
\(\sigma(\omega\cdot \zeta)\), \(q(\zeta)\),计算 \(z_H(\zeta)\),
\(L_0(\zeta)\), \(\sigma(\zeta)\) 与 \(id(\zeta)\);

验证步:Verifier 验证

\[
L_0(\zeta)(z(\zeta)-1) + \alpha\cdot (z(\omega\cdot \zeta)(b(\zeta)+\beta\cdot\sigma(\zeta)+\gamma)-z(\zeta)(a(\zeta)+\beta\cdot id(\zeta)+\gamma)) \overset{?}{=} q(\zeta)z_H(\zeta)
\]

协议完毕。

% \hypertarget{references}{%
% \subsection{References:}\label{references}}

% \begin{itemize}
% \tightlist
% \item
%   {[}WIP{]} Copy constraint for arbitrary number of wires.
%   https://hackmd.io/CfFCbA0TTJ6X08vHg0-9\_g
% \item
%   Alin Tomescu. Feist-Khovratovich technique for computing KZG proofs
%   fast.
%   https://alinush.github.io/2021/06/17/Feist-Khovratovich-technique-for-computing-KZG-proofs-fast.html\#fn:FK20
% \item
%   Ariel Gabizon. Multiset checks in PLONK and Plookup.
%   https://hackmd.io/@arielg/ByFgSDA7D
% \end{itemize}


